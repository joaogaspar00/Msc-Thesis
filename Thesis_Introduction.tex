%%%%%%%%%%%%%%%%%%%%%%%%%%%%%%%%%%%%%%%%%%%%%%%%%%%%%%%%%%%%%%%%%%%%%%%%
%                                                                      %
%     File: Thesis_Introduction.tex                                    %
%     Tex Master: Thesis.tex                                           %
%                                                                      %
%     Author: Andre C. Marta                                           %
%     Last modified :  4 Mar 2024                                      %
%                                                                      %
%%%%%%%%%%%%%%%%%%%%%%%%%%%%%%%%%%%%%%%%%%%%%%%%%%%%%%%%%%%%%%%%%%%%%%%%

\chapter{Introduction}
\label{chapter:introduction}


%%%%%%%%%%%%%%%%%%%%%%%%%%%%%%%%%%%%%%%%%%%%%%%%%%%%%%%%%%%%%%%%%%%%%%%%
\section{Motivation}
\label{section:motivation}

testeee


%%%%%%%%%%%%%%%%%%%%%%%%%%%%%%%%%%%%%%%%%%%%%%%%%%%%%%%%%%%%%%%%%%%%%%%%
\section{Topic Overview}
\label{section:topic_overview}

When it comes to helicopter emergency manoeuvres, autorotation is one of the most important skills that pilots use in the event of a failure of the engine \cite{federal_aviation_administration_helicopter_2021}. This method uses a controlled fall that starts with reducing the collective pitch control. Then it uses upward airflow to provide lift, maintain rotation of the rotor blade, and postpone descent. To maintain control throughout the manoeuvre, the pilots adjust the direction and attitude of the helicopter using cyclic control. The pilot performs a flare, further slowing the descent velocity as the helicopter gets closer to Earth by lowering the collective a little. Because auto-rotation is a skill that is acquired via extensive training, it is a dependable way for pilots to land safely in the event of unplanned power outages, which greatly enhances aviation safety in general.

Comparably, more sophisticated technologies have replaced conventional parachute systems in the investigation of recovery systems in space missions. In particular, the \gls{vtc} recovery system poses development, installation, and cost issues despite providing precise control over the descent speed and trajectory \cite{federal_aviation_administration_helicopter_2021}. In light of these factors, a novel method using a rotary wing recovery system under autorotation effect appears in light of these factors. This method has been developed for years, but it has not been put into practice, probably because of financial and/or technological limitations.


By combining the benefits of controlled thrust vector and parachute systems, the suggested rotary wing recovery system promises maneuverability, economic feasibility, and reusability. Using autorotation, similar in a helicopter, the technology ensures a safe and controlled landing. This recovery approach is still a promising and unexplored area in the rapidly changing field of technical developments, and further research is necessary to fully realise its potential. The complex properties of the system and the phenomenon of auto-rotation serve as the main topics of current research.


%%%%%%%%%%%%%%%%%%%%%%%%%%%%%%%%%%%%%%%%%%%%%%%%%%%%%%%%%%%%%%%%%%%%%%%%
\section{Objectives}
\label{section:objectives}


Previous studies have been conducted to explore the use of auto-rotation phenomena as a recovery method for spacecraft vehicles. In this subject \gls{armada} project \cite{noauthor_armada_nodate} is a key publication once it was done by major players . Also, many studies  as \cite{steiner_rotary_nodate} take the first step by using \gls{armada} vehicle concepts for its studies.For instance, the master's thesis by Marques \cite{marques_design_2024} investigated the phenomena in axial flight, demonstrating through simulations that the system functions as expected. However, certain aerodynamic considerations and assumptions were made, as well as in other works as \textcolor{blue}{meter aqui mais referências de outros artigos com referencias a modelos simplificados}

The proposed works aims to dig deep into the concept, trying to create a more sophisticated and fidelity model for rotary wings under autorotation effect and increase the inside knowledge of the subject. So, for emproving and achieve the main goals, several other intermidiate objectives should be define. First objective is to conduct an literature review shall be made in order to understand concept's limitations and linking those limitations with modeling and wind tunnel tests limitations.

As second, a 6-D0F dynamic model is developed along with the \gls{bet} for the rotary wing aerodynamic analysis. This parts, highlight all the necessary mathematical tools to simulate and 

The third step is to build an simple Matlab simuator which 


%%%%%%%%%%%%%%%%%%%%%%%%%%%%%%%%%%%%%%%%%%%%%%%%%%%%%%%%%%%%%%%%%%%%%%%%
\section{Thesis Outline}
\label{section:outline}

The document is divided into three main parts which will focus on the three key points of this thesis: Literature Review, 6-DOF BET Model, and Simulations

The chapter \ref{chapter:literaturereview} reviews the state of the rocket recovery today, examining a range of pruposed models and project and comparing them with other system from simple parachutes to sophisticated propulsive landing systems. It starts from understading the major opportunities and the interest of new technologies in aerospace industrie, on section \ref{section:spacerace_reusability}. On the other hand, section \ref{section:autorotation_vehicles} focus on previous projects which try to use rotary-wing under the phenomena of autorotation for spacecraft recovery. In this section some topics and presented as vehicle concepts, mathemativcal models and wind tunnel test, and successes and difficulties of this projet. Focusing on autorotation, a crucial idea in aerodynamics, section \ref{section:autorotation_phenomena} examines the fundamental mathematical models behind it. The key point is to understand \textcolor{blue}{explicar}


Chapter \ref{chapter:mathematical_model} \textcolor{blue}{explicar}